\definecolor{myblue}{RGB}{33,84,157}
\usepackage{hyperref}
\hypersetup{colorlinks=true, linkcolor=myblue} 
\hypersetup{breaklinks=true}
\hypersetup{colorlinks=true, linkcolor=myblue} 
\setbeamercolor{progress bar}{fg=myblue}
\setbeamercolor*{structure}{bg=myblue!20,fg=myblue}
\setbeamercolor*{palette primary}{use=structure,fg=white,bg=structure.fg}
\setbeamercolor*{palette secondary}{use=structure,fg=white,bg=structure.fg!75}
\setbeamercolor*{palette tertiary}{use=structure,fg=white,bg=structure.fg!50!black}
\setbeamercolor*{palette quaternary}{fg=white,bg=black}
\setbeamercolor{section in toc}{fg=black,bg=white}
\setbeamercolor{alerted text}{use=structure,fg=structure.fg!50!black!80!black}
\setbeamercolor{titlelike}{parent=palette primary,fg=structure.bg!50!black}
\setbeamercolor{frametitle}{bg=myblue!10,fg=myblue}
\setbeamertemplate{itemize items}[ball]
\setbeamercolor{block title}{bg=myblue!20!white,fg=myblue}
\setbeamercolor{block body}{use=structure,fg=black,bg=myblue!10}
\setbeamertemplate{theorems}[numbered]
\setbeamertemplate{caption}[numbered]
\setbeamertemplate{footline}[frame number]
\setbeamertemplate{navigation symbols}{\insertbackfindforwardnavigationsymbol}
\usepackage{appendixnumberbeamer}
\setbeamercolor{background canvas}{bg=white}
\setbeamertemplate{section in toc}[square]
\setbeamertemplate{subsection in toc}[square]
\usefonttheme{serif}
\usepackage{amsmath,amsfonts,amssymb,amsthm}
\usepackage{color}
\usepackage{graphicx, multirow}
\usepackage{epsfig}
\usepackage{comment}
\usepackage{caption}
\usepackage{romannum}
\usepackage{constants}
\usepackage{kotex}
\usepackage{subfigure}
\usepackage{bm}
\usepackage{booktabs}
\usepackage{multicol}
\usepackage{listings}
\usepackage{tikz}
\usetikzlibrary{shapes,decorations,arrows,calc,arrows.meta,fit,positioning,graphs}
\tikzset{
	-Latex,auto,node distance=1.5cm and 1.3cm, thick,
	state/.style={ellipse, draw, minimum width=0.9cm},
	point/.style={circle, draw, inner sep=0.18cm, fill, node contents={}},
	bidirected/.style={Latex-Latex,dashed},
	el/.style={inner sep=2.5pt, align=right, sloped}}
\usepackage[export]{adjustbox}[2011/08/13]
\makeatletter
\renewcommand{\metropolis@enablesectionpage}{
	\AtBeginSection{
		\ifbeamer@inframe
		\sectionpage
		\else
		\frame[c]{\sectionpage}
		\fi
	}
}
\metropolis@enablesectionpage
\makeatother
\usepackage{xcolor,listings}
\lstset{upquote=true}
\lstset{breaklines=true} 
\lstset{breakatwhitespace=false}
\captionsetup{labelformat=empty}
\theoremstyle{plain}
\newtheorem{corrolary}{Corollary}
\newtheorem{propposition}{Proposition}
\newtheorem{lemmma}{Lemma}
\theoremstyle{definition}
\newtheorem{assumption}{Assumption}
%------------------------------------------------------------------------------%
\setbeamertemplate{section in toc}[square]
\setbeamertemplate{subsection in toc}[square]

\AtBeginSection[] {
	\begin{frame}<beamer>
		\frametitle{Outline}
		\tableofcontents[currentsection]
	\end{frame}
}

\AtBeginSubsection[] {
	\begin{frame}<beamer>
		\frametitle{Outline}
		\tableofcontents[currentsection,hideothersubsections,currentsubsection]
	\end{frame}
}
%------------------------------------------------------------------------------%
\newcommand{\BR}{{\\ \medskip}}
\newcommand{\BM}{\begin{bmatrix}}
\newcommand{\EM}{\end{bmatrix}}
%\newcommand{\av}[1]{{\left| #1 \right|}}
\newcommand{\E}{\mb{E}}
\newcommand{\pr}{\mb{P}}
\newcommand{\bs}{{\overline s}}
\DeclareMathOperator*{\var}{Var}
\DeclareMathOperator*{\VAR}{Var}
\DeclareMathOperator*{\cov}{cov}
\DeclareMathOperator*{\COV}{Cov}
\DeclareMathOperator*{\cor}{cor}
\newcommand{\indep}{\perp \!\!\! \perp}
%\newcommand{\ip}[2]{{\left\langle #1, #2 \right\rangle}}
%\newcommand{\set}[1]{{\left\{ #1 \right\}}}
%\newcommand{\pa}[1]{{\left( #1 \right)}}
%\newcommand{\br}[1]{{\left[ #1 \right]}}
%\newcommand{\tr}[1]{{\ms{tr}\!\left[ #1 \right]}}
%\newcommand{\norm}[1]{{\lVert #1 \rVert}}
%\newcommand{\Norm}[1]{{\left\lVert #1 \right\rVert}}
\newcommand{\DP}{{\,\Bigdot\,}}
\newcommand{\mc}[1]{\mathcal{#1}}
\newcommand{\mb}[1]{\mathbb{#1}}
\newcommand{\mf}[1]{\mathfrak{#1}}
\newcommand{\ms}[1]{\mathsf{#1}}
\newcommand{\ba}{{\bm \alpha}}
\newcommand{\bb}{{\bm \beta}}
\newcommand{\be}{{\bm \varepsilon}}
\newcommand{\va}{{\bm a}}
\newcommand{\vb}{{\bm b}}
\newcommand{\vc}{{\bm c}}
\newcommand{\ve}{{\bm e}}
\newcommand{\vx}{{\bm x}}
\newcommand{\vy}{{\bm y}}
\newcommand{\vX}{{\mathbf X}}
\newcommand{\vY}{{\mathbf Y}}
\newcommand{\vC}{{\mathbf C}}
\newcommand{\e}{\varepsilon}
\newcommand{\R}{\mb{R}}
\newcommand{\AND}{{\quad \mbox{and} \quad}}
\newcommand{\for}{{\quad \mbox{for} \quad}}
\newcommand{\blue}[1]{\textcolor{blue}{#1}}
\newcommand{\red}[1]{\textcolor{red}{#1}}
\newcommand{\emp}[1]{\textcolor{DarkRed}{#1}}
\newcommand{\DB}[1]{\textcolor{DarkBlue}{#1}}
\newcommand{\MB}[1]{\textcolor{myblue}{#1}}
\newcommand{\DC}[1]{\textcolor{DarkCyan}{#1}}
\newcommand{\DR}[1]{\textcolor{DarkRed}{#1}}
\newcommand{\vo}{\vspace{0.1cm}}
\newcommand{\vt}{\vspace{0.25cm}}
\DeclareMathOperator*{\argmax}{argmax}
\DeclareMathOperator*{\argmin}{argmin}
\DeclareMathOperator*{\MISE}{MISE}
\DeclareMathOperator*{\SSE}{SSE}
\DeclareMathOperator*{\SST}{SST}
\DeclareMathOperator*{\SSR}{SSR}
\DeclareMathOperator*{\MSE}{MSE}
\DeclareMathOperator*{\MSR}{MSR}
\DeclareMathOperator{\df}{df}
\newcommand{\se}{\text{s.e.}}
\DeclareSymbolFont{cmdit}{OML}{cmr}{m}{it}
\DeclareMathSymbol{Y}{\mathalpha}{cmdit}{`Y}
\renewcommand{\baselinestretch}{1.3}
\newconstantfamily{FixedConstants}{symbol={c}}
\newcommand{\M}{\ensuremath{\C[FixedConstants]}}
%\newcommand{\Ml}[1]{\ensuremath{\Cl[FixedConstants]{#1}}}
%\newcommand{\Mr}[1]{\ensuremath{\Cr{#1}}}
\usetheme{metropolis}
